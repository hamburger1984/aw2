%---------------------------------------------------------------------------------------------------
% Einführung
%---------------------------------------------------------------------------------------------------
\newpage
%\part{Anfang}
%\abstractentry{Kurzzusammenfassung}{KURZE BIS 5 zeilige Zusammenfassung}

\section{Einführung}

Die Bedeutung des Internets nimmt in den letzten Jahren stetig zu.
Kommunikation, Verwaltung und Abwicklung von Geschäften verlagern sich zunehmend ins Internet.
Gleichzeitig wachsen die Bestrebungen seitens der Politik und Unternehmen Einfluss auszuüben - sei es zur Durchsetzung von Gesetzen oder zur Gewinnmaximierung.
Angesichts dieser Tatsache ist es interessant zu analysieren, welche Akteure die Daten beispielsweise auf dem Weg zwischen Geschäftspartnern oder zwischen einer Behörde und Bürgern traversieren. %Da das Internet diese Informationen jedoch nicht "`freiwillig"' .. hmm ne..
Um dieses Wissen zu erlangen, sind Topologieanalysen nötig, wie sie das Routing-Atlas Projekt durchführt.

In dieser Ausarbeitung wird der Routing-Atlas sowie die geplanten Änderungen vorgestellt.

In Abschnitt~\ref{sec:atlas} wird zunächst auf die bereits vorhandenen Schritte und Datenquellen zur Erstellung des Routing-Atlas eingegangen.
Abschnitt~\ref{sec:zielsetzung} umreißt die Zielsetzung, deren Hauptbestandteil - Ersatz einer externen, nicht mehr aktualisierten Datenquelle - in Abschnitt~\ref{sec:konzept} erläutert wird.
Abschnitt~\ref{sec:challenge} zeigt, welche Herausforderungen erwartet werden.
Den Abschluss bilden der Ausblick auf weitere Ausbaumöglichkeiten des Routing-Atlas in Abschnitt~\ref{sec:ausblick} und die Zusammenfassung in Abschnitt~\ref{sec:schluss}

%Der Routing-Atlas ist ein Projekt der inet AG in Zusammenarbeit mit dem BSI.
%Ziel des Projektes ist es landesspezifische Teile des Internet und Abhängigkeiten zwischen Ländern zu identifizieren.

%Zunächst werden IP-Adressblöcke, die zum betrachteten Land gehören, identifiziert.
%Die gefundenen Adressblöcke werden zu routbaren Präfixen und anschließend autonomen Systemen (ASe) zusammengefasst.
%Mittels einer am NEC-lab erstellten Matrix wird ein Graph der landesinternen und verbindenden ASe gebildet.
%Diese Matrix wird jedoch nicht mehr aktualisiert, sodass Ersatz geschaffen werden soll.
%Dies (und die interaktive Darstellung der Ergebnisse für Interessierte) ist das Ziel des Autors.


