%---------------------------------------------------------------------------------------------------
% Einführung
%---------------------------------------------------------------------------------------------------
\newpage
%\part{Anfang}
%\abstractentry{Kurzzusammenfassung}{KURZE BIS 5 zeilige Zusammenfassung}

\section{Einführung}

Die Bedeutung des Internets nimmt in den letzten Jahren stetig zu.
Kommunikation, Verwaltung und Abwicklung von Geschäften verlagern sich zunehmend ins Internet.
Gleichzeitig wachsen die Bestrebungen seitens der Politik und Unternehmen Einfluss auszuüben - sei es zur Durchsetzung von Gesetzen oder zur Gewinnmaximierung.
Angesichts dieser Tatsache ist es interessant zu analysieren, welche Akteure die Daten beispielsweise auf dem Weg zwischen Geschäftspartnern oder zwischen einer Behörde und Bürgern traversieren. %Da das Internet diese Informationen jedoch nicht "`freiwillig"' .. hmm ne..
Um dieses Wissen zu erlangen, sind Topologieanalysen nötig, wie sie das Routing-Atlas Projekt durchführt.

Der Bereich Topologieanalyse wird von weiteren Arbeitsgruppen bearbeitet.
Teils werden die dabei entstehenden Resultate vom Routing-Atlas Projekt direkt genutzt, teils werden eigene Wege beschritten.
In der vorliegenden Ausarbeitung wird eine Auswahl vergleichbarer Arbeiten vorgestellt.
Dabei wird grob zwischen den Bereichen Datensammlung und Modellierung unterschieden.
Abschnitt~\ref{sec:datenquellen} stellt Arbeiten im Bereich Datensammlung vor, Abschnitt~\ref{sec:modellierung} Arbeiten im Bereich Modellierung.

% In dieser Ausarbeitung wird der Routing-Atlas sowie die geplanten Änderungen vorgestellt.

% In Abschnitt~\ref{sec:atlas} wird zunächst auf die bereits vorhandenen Schritte und Datenquellen zur Erstellung des Routing-Atlas eingegangen.
% Abschnitt~\ref{sec:zielsetzung} umreißt die Zielsetzung, deren Hauptbestandteil - Ersatz einer externen, nicht mehr aktualisierten Datenquelle - in Abschnitt~\ref{sec:konzept} erläutert wird.
% Abschnitt~\ref{sec:challenge} zeigt, welche Herausforderungen erwartet werden.
% Den Abschluss bilden der Ausblick auf weitere Ausbaumöglichkeiten des Routing-Atlas in Abschnitt~\ref{sec:ausblick} und die Zusammenfassung in Abschnitt~\ref{sec:schluss}

