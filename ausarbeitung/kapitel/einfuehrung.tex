%---------------------------------------------------------------------------------------------------
% Einführung
%---------------------------------------------------------------------------------------------------
\newpage

\section{Einführung}

Die Bedeutung der Internets für Wirtschaft, Bildung und Staat nimmt zu.
Angesichts dieser Entwicklung hat sich das Internet von einer Spielwiese für Technikbegeisterte und Forschung zu einem Stück Infrastruktur gewandelt.
Als solche ist ein Verständnis des Aufbaus und die Analyse der Topologie wichtig.
Zum einen um Schwachstellen der Vernetzung zu entdecken.
Zum anderen um ein Verständnis für Beteiligte an der Datenvermittlung und damit für potentielle Einflussnahmen zu erlangen.
Das Routing-Atlas Projekt führt derartige Analysen am Beispiel Deutschland durch.

Der Bereich Topologieanalyse wird von weiteren Arbeitsgruppen bearbeitet.
Teils werden die dabei entstehenden Resultate vom Routing-Atlas Projekt direkt genutzt, teils werden eigene Wege beschritten.
In der vorliegenden Ausarbeitung wird in Abschnitt~\ref{sec:routingatlas} zunächst das Routing-Atlas Projekt vorgestellt.
Im Anschluss wird eine Auswahl vergleichbarer Arbeiten vorgestellt.
Dabei wird grob zwischen den Bereichen Datensammlung und Modellierung unterschieden.
Abschnitt~\ref{sec:datenquellen} stellt Arbeiten im Bereich passiven und aktiven Datensammlung vor, Abschnitt~\ref{sec:modellierung} Arbeiten im Bereich Modellierung.
Der Abschnitt~\ref{sec:ausblick} gibt einen Ausblick auf zukünftige Aufgaben und Entwicklungen im Kontext des Routing-Atlas, Abschnitt~\ref{sec:schluss} fasst die vorliegende Ausarbeitung zusammen.
