\section{Ausblick}\label{sec:ausblick}

Mit einer neu berechneten shortest path und next hop matrix wird die Aktualität der Datenbasis des Routing-Atlas vorerst wiederhergestellt.
Darauf basierend sollen die Bestandteile des Routing-Atlas soweit automatisiert werden, dass sie regelmäßig ausgeführt und Veränderungen über die Zeit dargestellt werden können.
Diese Darstellungen sollen in Form eines Online-Dienstes bereitgestellt werden.

Um vergleichende Analysen zwischen verschiedenen Ländern zu ermöglichen, sollen die Methoden des Routing-Atlas auf andere (europäische) Länder angewendet werden.
Der Fokus liegt hierbei auf Europa, weil dies zum einen die nähere Umgebung betrifft und zum anderen die Qualität der Daten der europäischen RIR (RIPE NCC) gegenüber denen anderer RIRs gut ist.

Weiterhin besteht der Plan, die momentan auf das IPv4-Protokoll bezogenen Methoden des Routing-Atlas auf die Anwendung auf IPv6 zu übertragen.
Hierbei ist der Flaschenhals derzeit die auf IPv4 begrenzte Datensammlung des IRL.
Routekollektoren und RIPE DB unterstützen IPv6 bereits.
Die Daten des IRL sind eine wertvolle Quelle für im Rahmen des Routing-Atlas durchgeführten Topologieanalysen.
Es wäre also vor der Portierung des Routing-Atlas durchaus sinnvoll die (hoffentlich baldige) Integration der Beobachtung von IPv6 seitens des IRL abzuwarten.
Der Aufwand auch für diese Sammlung einen Ersatz zu schaffen dürfte relativ hoch sein.

