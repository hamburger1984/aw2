\section{Modellierung}\label{sec:modellierung}

Alleine aus der Existenz eines Links zwischen ASen lässt sich nicht abschätzen, ob und für welchen Datenverkehr der Link benutzt wird.
Es erfordert vielmehr die Bildung eines Modells, das das policy based routing von BGP nachbildet.
Ein solches Modell klassifiziert die Art der Beziehung zwischen zwei ASen.
Basierend auf dieser Klassifizierung kann hergeleitet werden, welche Pfade möglich sind und verwendet werden.
Dabei wird von einem weitesgehend hierarchischem Aufbau AS Topologie ausgegangen.
Die zentraleren Transit oder Tier-1 ASe leiten hierbei Daten für die kleineren ASe weiter.
Kleinere ASe bieten dagegen keinen Transit für größere ASe an.
Dieses Konzept bezeichnet man als das Valley-free Routing.
Über ein Modell sollen rohe Daten wie BGP Tabellen um Klassifizierungen erweitert werden um auf Basis des so entstehenden attributierten AS-Graphen Analysen ausführen zu können.

\subsection{On Inferring Autonomous System Relationships in the Internet}
Grundlegende und viel beachtete Arbeit im Bereich der Klassifizierung von AS Beziehungen hat Lixin Gao (University of Massachusetts, damals AT\&T Research Labs) mit der im folgenden Abschnitt vorgestellten Arbeit geleistet~\cite{Gao:2001:IAS:504611.504616}.
Das Prinzip des valley-free Routings führt zu der Erkenntnis, dass Verbindung im Internet nicht Erreichbarkeit bedeuten muss.
Der Transport von Daten folgt policies, die über das BGP Protokoll implementiert werden.
Ziel der Arbeit von Lixin Gao ist es die Kanten eines AS Graphen zu annotieren, also die Links zu klassifizieren.
Die Klassen sind dabei:
\begin{itemize}
  \item customer-provider
  \item peering
  \item sibling
\end{itemize}
Da die Geschäftsbeziehung mehrerer ASe Ursache für die Gestaltung von BGP tables, nicht aber deren Bestandteil sind, wird diese Information über eine Heuristik hergeleitet.

\subsubsection{Vorgehensweise}
Für eine Menge von BGP Tabellen wird der Verzweigungsgrad der ASe bestimmt, der später als Indiz für die Größe des ASes verwendet wird.
In jedem AS Pfad in den BGP Tabellen wird nun das größte AS gesucht.
Die Links im Pfad bis zum größten AS (uphill) werden als Nutzer von Transit, die Links ab dem größten AS (downhill) als Anbieter von Transit markiert.
In einer weiteren Phase geschieht nun die eigentliche Klassifizierung.
Ist ein Paar aus ASen in beide Richtungen als Transit markiert, so gehört dieser Link in die Klasse "`sibling"'.
Ist ein Paar aus ASen Nutzer von Transit aber selten oder nie Anbieter, ist der Link ein "`customer-provider"'-Link, überwiegt die Anbieterrolle "`provider-customer"'.
Die so hergeleiteten Klassifizierungen der AS Beziehungen wurden mit intern vorhandenen Informationen von AT\&T abgeglichen wobei eine Übereinstimmung von 99,1\% festgestellt wurde.

\subsubsection{Beitrag}
Lixin Gao hat mit ihrer Arbeit grundlegende Ideen zur Herleitung einer Klassifizierung von AS Beziehungen geprägt.

\subsection{Modeling the Internet Routing Topology - in less than 24h}\label{subsec:winter}
Der attributierte AS-Graph liefert Informationen über Existenz und Art der Verbindung der ASe.
Der Weg, den der Datenpakete zwischen zwei ASen traversiert, lässt sich hieraus jedoch nicht ablesen.
Rolf Winter (damals NEC Labs Europe, jetzt Hochschule Augsburg) beschreibt in seinem Artikel "`Modeling the Internet Routing Topology - in less than 24h"'~\cite{conf/pads/Winter09} einen Ansatz, die kürzesten Wege zwischen allen AS-Paaren unter Berücksichtigung eines Modells von BGP zu berechnen.
Die Ergebnisse der Arbeit sind unter \url{http://topology.neclab.eu/} veröffentlicht.

\subsubsection{Vorgehensweise}
Winter nutzt die Daten des in Abschnitt~\ref{subsec:collecting} vorgestellten Projekts der UCLA als Basisdaten seiner Analyse.
Aus diesen Daten sollen innerhalb der schon im Titel erwähnten maximal 24 Stunden die kürzesten Pfade für alle AS Paare errechnet werden.
Die Betrachtung der Internettopologie auf Ebene der ASe stellt eine Vereinfachung dar.
Das Routing findet tatsächlich auf Ebene der Präfixe statt.
Laut Winter ist diese Vereinfachung aber zulässig, da der überwiegende Teil der Pfade zwischen zwei ASen unabhängig vom Quell- oder Zielpräfix identisch sind und somit kein signifikanter Verlust an Information zu erwarten ist.
Als weitere Rechtfertigung für die Beschränkung auf die AS Ebene dient die Zeitvorgabe.
Die Berechnung der kürzesten Pfade auf Ebene der Präfixe wäre komplexer und würde so die zeitnahe Bereitstellung aktueller Pfadmatrizen verhindern.
Das all-pair shortest path Problem - also die Berechnung der kürzesten Wege - löst Winter mit dem Floyd-Warshall Algorithmus.
Die Kantengewichte für die Modellierung der BGP policies basieren hierbei auf der relativen Häufigkeit, mit der die vorhandenen ausgehenden Links eines AS in beobachteten Routingtabellen verwendet werden.

In einem weiteren Schritt werden BGP Tabellen von RouteViews und RIPE RIS in unverarbeiteter Form zur Validierung der errechneten Pfade genutzt.
Bei erfolgreicher Nachbildung des policy based routings in BGP müssten die errechneten Pfade in den BGP Tabellen auftauchen.
Winter ermittelt die Übereinstimmungen zwischen Realtität und Modell für Pfadlänge und Pfadüberlappung.
Mit knapp 75\% Wahrscheinlichkeit unterscheiden sich die reale und errechnete Länge der Pfade nicht (sind also exakte Übereinstimmungen), mit knapp 95\% Wahrscheinlichkeit beträgt der Längenunterschied maximal eins, mit nahezu 100\% Wahrscheinlichkeit maximal zwei.
Die Überlappung von errechnetem und realem Pfad ist in knapp 45\% der Fälle komplett gegeben, mehr als 80\% der errechneten Pfade stimmen zu mehr als 50\% mit realen Pfaden überein.
Es bleiben aber über 15\% der errechneten Pfade, die mit keinem in den Validierungsdaten existierendem Pfad übereinstimmen.

\subsubsection{Beitrag}
Winter hat einen Ansatz präsentiert, basierend auf existierenden Daten zu vorhandenen AS Links kürzeste Pfade unter Berücksichtigung des policy based routings zu modellieren.
Die dabei errechneten Pfade wurden öffentlich zur Verfügung gestellt und für die Dauer der Förderung des Projektes regelmäßig aktualisiert.
Leider ist das Projekt mittlerweile mangels finanzieller Unterstützung eingestellt, sodass für das Routing Atlas Projekt ein Ersatz geschaffen werden muss.
