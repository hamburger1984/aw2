\section{Modellierung}\label{sec:modellierung}

Alleine aus der Existenz eines Links zwischen ASen lässt sich nicht abschätzen, ob und für welchen Datenverkehr der Link benutzt wird.
Es erfordert vielmehr die Bildung eines Modells, das das policy based routing von BGP nachbildet.
Ein solches Modell klassifiziert die Art der Beziehung zwischen zwei ASen.
Basierend auf dieser Klassifizierung kann hergeleitet werden, welche Pfade möglich sind und verwendet werden.
Dabei wird von einem weitesgehend hierarchischem Aufbau AS Topologie ausgegangen.
Die zentraleren Transit oder Tier-1 ASe leiten hierbei Daten für die kleineren ASe weiter.
Kleinere ASe bieten dagegen keinen Transit für größere ASe an.
Dieses Konzept bezeichnet man als das Valley-free Routing.
Über ein Modell sollen rohe Daten wie BGP Tabellen um Klassifizierungen erweitert werden um auf Basis des so entstehenden attributierten AS-Graphen Analysen ausführen zu können.

\subsection{On Inferring Autonomous System Relationships in the Internet}
Grundlegende und viel beachtete Arbeit im Bereich der Klassifizierung von AS Beziehungen hat Lixin Gao mit der im folgenden Abschnitt vorgestellten Arbeit geleistet~\cite{Gao:2001:IAS:504611.504616}.
Das Prinzip des valley-free Routings führt zu der Erkenntnis, dass Verbindung im Internet nicht Erreichbarkeit bedeuten muss.
Der Transport von Daten folgt policies, die über das BGP Protokoll implementiert werden.
Ziel der Arbeit von Lixin Gao ist es die Kanten eines AS Graphen zu annotieren, also die Links zu klassifizieren.
Die Klassen sind dabei:
\begin{itemize}
  \item customer-provider
  \item peering
  \item sibling
\end{itemize}
Da die Geschäftsbeziehung mehrerer ASe Ursache für die Gestaltung von BGP tables, nicht aber deren Bestandteil sind, wird diese Information über eine Heuristik hergeleitet.

\subsubsection{Vorgehensweise}
Für eine Menge von BGP Tabellen wird der Verzweigungsgrad der ASe bestimmt, der später als Indiz für die Größe des ASes verwendet wird.
In jedem AS Pfad in den BGP Tabellen wird nun das größte AS gesucht.
Die Links im Pfad bis zum größten AS (uphill) werden als Nutzer von Transit, die Links ab dem größten AS (downhill) als Anbieter von Transit markiert.
In einer weiteren Phase geschieht nun die eigentliche Klassifizierung.
Ist ein Paar aus ASen in beide Richtungen als Transit markiert, so gehört dieser Link in die Klasse "`sibling"'.
Ist ein Paar aus ASen Nutzer von Transit aber selten oder nie Anbieter, ist der Link ein "`customer-provider"'-Link, überwiegt die Anbieterrolle "`provider-customer"'.

\subsubsection{Beitrag}

\subsection{Modeling the Internet Routing Topology - in less than 24h}\label{subsec:winter}
Der attributierte AS-Graph liefert Informationen über Existenz und Art der Verbindung der ASe.
Der Weg, den der Datenpakete zwischen zwei ASen traversiert, lässt sich hieraus jedoch nicht ablesen.
Rolf Winter (damals NEC Labs Europe, jetzt Hochschule Augsburg) beschreibt in seinem Artikel "`Modeling the Internet Routing Topology - in less than 24h"'~\cite{conf/pads/Winter09} einen Ansatz, die kürzesten Wege zwischen allen AS-Paaren unter Berücksichtigung eines Modells von BGP zu berechnen.

\subsubsection{Vorgehensweise}
....


\subsubsection{Beitrag}
....
