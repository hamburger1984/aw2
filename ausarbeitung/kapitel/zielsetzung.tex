% \section{Zielsetzung}\label{sec:zielsetzung}
%
% Ziel des Autors ist die Erweiterung des Routing-Atlas.
% Die Erweiterungen lassen sich in die akut nötige Schaffung eines Ersatz für die next hop matrix des NEC-Lab und weniger dringende, dennoch interessante und erstrebenswerte Weiterentwicklungen unterteilen.
% Das NEC-Lab Projekt, das die in Abschnitt \ref{atlas:graph} vorgestellte next hop matrix erstellt, wurde zum Ende des Jahres 2009 eingestellt.
% Daraus folgt, dass eine wichtige Datengrundlage des Routing-Atlas seither altert.
% Die Topologie des Internet ist stetigen Änderungen unterworfen, die so vom Routing-Atlas nicht erfasst wird.
%
% Primäre Zielsetzung der Erweiterungen am Routing-Atlas ist es einen kontinuierlich aktualisierten Ersatz für die next hop matrix des NEC-Lab zu schaffen. Ein Konzept hierfür wird in Kapitel~\ref{sec:konzept} vorgestellt.
%
% Daneben sind weitere Entwicklungen geplant:
% \begin{description}
%  \item[Online-Aktualisierung] Um Veränderungen der Topologie und deren Auswirkungen besser beobachten zu können, sollte der Routing-Atlas automatisiert regelmäßig neu generiert und die Ergebnisse - in einem zu entwerfenden interaktiven Werkzeug - dargestellt werden.
%  \item[Weitere Länder] Die Anwendung der Verfahren für die Generierung des deutschen Routing-Atlas auf ein anderes europäisches Land erfordert Anpassungen und ermöglicht vergleichende Analysen der Topologien verschiedener Länder.
%  \item[IPv6] Die "`Portierung"' des Routing-Atlas in die IPv6-Welt ist aufgrund der Aktualität dieses Themenbereichs und der hier stattfindenden Entwicklung und Bewegung interessant. Hier stellt sich die Herausforderung, dass die vom IRL der UCLA bereitgestellten Daten derzeit nur IPv4 berücksichtigen.
% \end{description}
