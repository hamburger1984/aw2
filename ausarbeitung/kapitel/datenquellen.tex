\section{Datenquellen}\label{sec:datenquellen}

Das Internet ist eine Menge untereinander verbundener Netze, den autonomen Systemen (ASe).
Die Verwaltung dieser Netze obliegt jeweils dem Besitzer. Es gibt nicht eine zentrale Verwaltung, die über sämtliche Informationen verfügt.
Diese Aufgabe erfüllen lokale Registries (RIRs), bei denen beispielsweise eine ASN beantragt werden kann.
Die Existenz eines AS und einige Informationen zu einem AS lassen sich bei den RIRs prüfen.
Die Existenz und Konditionen der Verbindung zweier ASe wird nicht erfasst, sondern lässt sich aus aktiven Messungen und der Auswertung von Routingtabellen mehr oder weniger präzise herleiten.
Die Information, ob ASe miteinander verbunden sind und welcher Art diese Verbindung ist ermöglicht die Berechnung der Topologie des Internets auf AS Ebene.

Als Beispiel für das passive Methoden der Datensammlung dient hier die Arbeit von Lixia Zhang et\ al., für aktive Methoden der Beitrag von Brice Augusting et\ al..

\subsection{Collecting the Internet AS-level Topology~\cite{Zhang:2005:CIA:1052812.1052825}}

Beichuan Zhang, Raymond Liu, Lixia Zhang (alle UCLA) und Daniel Massey (Colorado State University) veröffentlichten im Januar 2005 den Artikel "`Collecting the Internet AS-level Topology"'.
Darin beschreiben sie die Schaffung einer gegenüber vorher existierenden Ansätzen vollständigeren Datenbasis für Topologieanalysen auf AS-Ebene.

\subsubsection{Übersicht}
Die Gruppe um Lixia Zhang ergänzt die üblicherweise verwendeten Topologiedaten des Route Views Projekts\footnote{\url{http://www.routeviews.org/}} und des RIPE RIS\footnote{\url{http://ripe.net/ris}} und stellt diese täglich aktualisiert unter \url{http://irl.cs.ucla.edu/topology/} zur Verfügung.
Weiterhin basierten Topologieanalysen bisher meist auf statischen Schnappschüssen zu einem bestimmten Zeitpunkt.
Zwischen ASen existieren jedoch meist mehrere verbindende Pfade, wovon der sekundäre erst bei Ausfall des primären sichbar werden kann.
Um derartige Konstellationen zu erfassen, werden die Routinginformationen kontinuierlich gesammelt und auf Linkebene mit weiteren Informationen (Zeitpunkt \& Quelle der Beobachtung) versehen.
So werden Analysen über die Stabilität der Topologie und Änderungsraten von Pfaden möglich.

\subsubsection{Verwendete Datenquellen}
Route Views und RIPE RIS sammeln ihre Topologiedaten mittels \emph{BGP trace collector}.
Dies ist ein PC oder Router, der von ISPs BGP Nachrichten erhält und speichert, jedoch selber keine Routen bekannt gibt.
Routingtabellen und -updates werden in regelmäßigen Abständen abgeholt und veröffentlicht.
Die Berücksichtigung der Routingupdates führt hierbei dazu, dass auch nicht aktuell für das Routing verwendete Pfade erfasst werden.\\

Ergänzt werden diese Daten um Routingtabellen öffentlich zugänglicher \emph{Routingserver} einiger ISPs.
Die Routingtabellen enthalten weitere verwendete Pfade.
Alternativrouten können aufgrund dieser Datenquelle nur durch die Beobachtung über einen längeren Zeitraum gefunden werden.
Weiterhin speichert ein Routingserver keine Historie der Routingtabellen.
Es muss also in regelmäßigen Intervallen von außen der Dump der Routingtabelle angefordert und dort gespeichert werden.\\

\emph{Looking glasses} sind Server, die über eine Weboberfläche die Ausführung bestimmter Befehle auf einem Router ermöglichen.
Vollständige Routingtabellen lassen sich hier meist nicht abfragen aber immerhin direkte Nachbarn und Zugehörigkeit zu einem AS.
So liefert ein Looking glas Server jeweils einige Links (zu seinen Nachbarn), jedoch keine längeren Pfade (wie sie in einer Routingtabelle stehen würden).



\subsection{IXPs: Mapped?~\cite{Augustin:2009:IM:1644893.1644934}}

Wollen Peerings an IXPs finden, indem sie traceroutes "durch" IXPs aufrufen.
