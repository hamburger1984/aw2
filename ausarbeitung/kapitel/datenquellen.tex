\section{Datenquellen}\label{sec:datenquellen}

Das Internet ist eine Menge untereinander verbundener Netze, den autonomen
Systemen (ASe). Die Verwaltung dieser Netze obliegt jeweils dem Besitzer. Es
gibt nicht eine zentrale Verwaltung, die über sämtliche Informationen verfügt.
Diese Aufgabe erfüllen lokale Registries (RIRs), bei denen beispielsweise eine
ASN beantragt werden kann. Die Existenz und einige Informationen zu einem AS
lassen sich bei den RIRs prüfen. Die Existenz und Konditionen der Verbindung
zweier ASe wird nicht erfasst, sondern lässt sich aus aktiven Messungen und der
Auswertung von Routingtabellen mehr oder weniger präzise herleiten. Die
Information, ob ASe miteinander verbunden sind und welcher Art diese Verbindung
ist ermöglicht die Berechnung der Topologie des Internets auf AS Ebene.

Als Beispiel für das passive Methoden der Datensammlung dient hier die Arbeit
von Beichuan Zhang et\ al., für aktive Methoden der Beitrag von Brice Augusting
et\ al..

\subsection{Collecting the Internet AS-level Topology~\cite{Zhang:2005:CIA:1052812.1052825}}

Sammeln Daten und stellen sie täglich aktualisiert bereit\footnote{\url{http://irl.cs.ucla.edu/topology/}}


\subsection{IXPs: Mapped?~\cite{Augustin:2009:IM:1644893.1644934}}

Wollen Peerings an IXPs finden, indem sie traceroutes "durch" IXPs aufrufen.
