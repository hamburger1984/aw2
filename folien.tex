\documentclass[ngerman,compress,hyperref={bookmarks}]{beamer}
\usetheme{Antibes}
\useoutertheme{infolines}
\usepackage[utf8x]{inputenc}

\usepackage{multirow}

\usepackage{colortbl}
\definecolor{dunkelgrau}{rgb}{0.8, 0.8, 0.8}

\usepackage{wasysym}

\logo{\includegraphics[height=1cm]{images/logoHAW}}
\usepackage{graphicx}
%\usepackage[%
%	bibstyle=authoryear,%
%	citestyle=authoryear,%
%	bibencoding=utf8,%
%	bibtex8=true,%
%	sorting=nyt,%
%	sortcites=true,%
%	maxnames=2,%
%	babel=other,%
%	block=space,%
%	backref=false,%
%	natbib=true,%
%	hyperref=true,%
%]{biblatex}
%\bibhang1em
%\usepackage[style=authortitle-icomp]{biblatex}
%\bibliography{routing_atlas}

%\setbeamertemplate{bibliography entry title}{}
%\setbeamertemplate{bibliography entry location}{}

\title{Erweiterung des Routing-Atlas}
\subtitle{Vortrag: Anwendung 2\\ Related Work}
\subject{Routing-Atlas, Topology}
\author{Andreas Krohn}
\institute[HAW]{Hochschule für Angewandte Wissenschaften Hamburg}
\date[SoSe 2012]{26. April 2012}

\begin{document}
\frame[plain]{\titlepage}

\section*{Agenda}
\begin{frame}{Agenda} \setcounter{tocdepth}{1} \tableofcontents[part=1] \setcounter{tocdepth}{3} \end{frame}

\part{Hauptteil}
\section{Rückblick}

\subsection{Was ist der Routing-Atlas?}
\begin{frame}{Was ist der Routing-Atlas?}
\nocite{wsbh-envgi-12}
  \begin{itemize}
    \item Projekt der inet AG und des BSI
    \item Topologieanalyse um landesspezifische Teile des Internets zu
    \begin{enumerate}
      \item Identifizieren
      \item Klassifizieren
      \item Visualisieren
    \end{enumerate}
  \end{itemize}
  \vspace{1cm}
  \begin{thebibliography}{}
    \bibitem{wsbh-envgi-12} ``Exposing a Nation-Centric View on the German Internet – A Change in Perspective on the AS Level''
    \newblock Wählisch, Matthias and Schmidt, Thomas C. and de Brün, Markus and Häberlen, Thomas (2012)\\[-20pt]
  \end{thebibliography}
\end{frame}

\subsection{Bestandteile des Routingatlas}
\begin{frame}{Bestandteile des Routingatlas}
  \begin{enumerate}
  \item Identifikation deutscher Autonomer Systeme
  \begin {itemize}
    \item IP-Blöcke identifizieren
    \item Zu IP-Präfixen auflösen
    \item IP-Päfixe Autonomen Systemen zuordnen
  \end{itemize}
  \item Neu dazugekommen: Validierung mittels Maxmind \& Cymru
  \item Klassifikation Autonomer Systeme
  \begin{itemize}
    \item Topologische Einordnung
    \item Branchen
  \end{itemize}
  \item Routing-Graphen bilden
  \begin{itemize}
    \item shortest path matrix des NEC-Lab
  \end{itemize}
  \item Visualisierung der (Teil)Graphen
  \end{enumerate}
\end{frame}

\subsection{Ziel}
%\begin{frame}<1-2>{Ziel}
\begin{frame}{Ziel}
  \begin{center}
    {\Large Erweiterung des Routing-Atlas}\\
    \vspace{0.5cm}
    shortest path matrix ersetzen\\
    \vspace{0.5cm}
    Dazu:
    \begin{itemize}
      \item Datenquellen (Routingtabellen, BGP Peering Informationen der IRR)
      %\item \only<1>{Heuristik zur Bewertung von Inter-AS Links}\only<2>{\textbf{Heuristik zur Bewertung von Inter-AS Links}}
      \item Heuristik zur Bewertung von Inter-AS Links
      %\item \only<1>{Kürzeste Wege berechnen}\only<2>{\textbf{Kürzeste Wege berechnen}}
      \item Kürzeste Wege berechnen
    \end{itemize}
  \end{center}
\end{frame}

\section*{Agenda}
\begin{frame}{Agenda} \setcounter{tocdepth}{1} \tableofcontents[part=1] \setcounter{tocdepth}{3} \end{frame}

\section{Begriffe, Grundlagen}
\begin{frame}{Begiffe, Grundlagen}
  \begin{columns}[c]
    \begin{column}{0.75\textwidth}
      \begin{itemize}
        \item Internet: Ansammlung Autonomer Systeme (ASe)
        \item Geschäftsbeziehung zwischen ASen $\rightarrow$ AS Link
      \end{itemize}
      \begin{description}
        \item[C2P] \textbf{C}ustomer bezahlt \textbf{P}rovider für Transit
        \item[PP] \textbf{P}eer-\textbf{P}eer - kostenneutralen Austausch von Traffic
      \end{description}
      \begin{itemize}
        \item Öffentlich verfügbar: BGP dumps, traceroutes, looking glass server
        \item \textbf{Aber} keine zentrale Verwaltung, Vermessungsanstalt, "Ground truth"
        \item Topologie aus vorhandenen Daten herleiten
      \end{itemize}
      %......... \\
      %kl. Grafik?
    \end{column}
  \end{columns}
\end{frame}

%\section{AS-Beziehungen, valley-free routing}
%\begin{frame}{AS-Beziehungen, valley-free routing}
% \begin{columns}[c]
%  \begin{column}{0.2\textwidth}
%  \end{column}
%  \begin{column}{0.5\textwidth}
%    \begin{enumerate}
%     \item[customer-provier] customer zahlt für Transit
%     \item[peering] kostenloser Trafficaustausch
%     \item[sibling] kostenloser Transit\\ \vspace{0.5cm}
%     \item[valley-free routing] Transit nur über ``höhere'' ASes
%    \end{enumerate}
%  \end{column}
%  \begin{column}{0.3\textwidth}
%   \begin{figure}
%    \label{asrelations}
%    \includegraphics[width=1\textwidth]{images/asrelation}
%   \end{figure}
%    {\scriptsize }
%  \end{column}
% \end{columns}
%\end{frame}

\section*{Agenda}
\begin{frame}{Agenda} \setcounter{tocdepth}{1} \tableofcontents[part=1] \setcounter{tocdepth}{3} \end{frame}

\section{Related Work}

\subsection{Topologiemodellierung}

\begin{frame}{On power-law relationships of the Internet topology}{1999, \cite{Faloutsos:1999:PRI:316194.316229}}
  \begin{columns}[c]
    \begin{column}{0.5\textwidth}
      \begin{itemize}
        \item Graphentheoretische Betrachtung der Inter- \& Intradomain Topologie
        \item Entdeckung exponentieller Zusammenhänge zwischen
        \begin{itemize}
          \item ausgehenden Links und Rang eines AS
          \item Häufigkeit und Anzahl ausgehender Links
          \item Eigen exponent?
        \end{itemize}
      \end{itemize}
    \end{column}
    \begin{column}{0.1\textwidth}
      \begin{figure}
        \includegraphics[width=1\textwidth]{images/faloutsos_m}\\
        \includegraphics[width=1\textwidth]{images/faloutsos_p}\\
        \includegraphics[width=1\textwidth]{images/faloutsos_c}
        \label{faloutsos}
      \end{figure}
    \end{column}
    \begin{column}{0.3\textwidth}
      {\scriptsize Micalis Faloutsos\\
      \vspace{0.1cm}
      University of California\\
      \vspace{0.8cm}
      Petros Faloutsos\\
      \vspace{0.1cm}
      University of Toronto\\
      \vspace{0.7cm}
      Christos Faloutsos\\
      \vspace{0.1cm}
      Carnegie Mellon University\\ }
    \end{column}
  \end{columns}
\end{frame}

\begin{frame}{On Inferring Autonomous System Relationships in the Internet}{2001, \cite{Gao:2001:IAS:504611.504616}}
  \begin{columns}[c]
    \begin{column}{0.5\textwidth}
      \begin{itemize}
        \item AS Pfade aus BGP Routing Tabellen
        \item Knotengrad als Heuristik
        \item Einordnung in customer-provider/peering/sibling
      \end{itemize}
    \end{column}
    \begin{column}{0.1\textwidth}
      \begin{figure}
        \includegraphics[width=1\textwidth]{images/gao}
        \label{gao}
      \end{figure}
    \end{column}
    \begin{column}{0.3\textwidth}
      {\scriptsize Lixin Gao\\
      \vspace{0.1cm}
      University of Massachusetts\\
      AT\&T Research Labs}
    \end{column}
  \end{columns}
\end{frame}

% \begin{frame}{Type of Relationship (ToR)}
%  \begin{columns}[c]
%   \begin{column}{0.5\textwidth}
%    \begin{itemize}
%     \item \cite{Subramanian:2001:CIH:894120, Di_Battista:2007:CTR:1279660.1279662}
%     \item ``Sichtweisen'' getrennt betrachten
%     \item Graphentheoretisches Problem:
%     \begin{itemize}
%      \item Kanten als \emph{uphill}, \emph{downhill} oder \emph{peering} markieren
%      \item Dabei valide Pfade erreichen
%     \end{itemize}
%     \item (Stark) NP-vollständig
%     \item Heuristik: AS Rank
%    \end{itemize}
%
%   \end{column}
%   \begin{column}{0.1\textwidth}
%
%   \end{column}
%   \begin{column}{0.3\textwidth}
%
%   \end{column}
%  \end{columns}
%
% \end{frame}


\begin{frame}{Collection the Internet AS-level Topology}{2005, \cite{Zhang:2005:CIA:1052812.1052825}}
  \begin{columns}[c]
    \begin{column}{0.5\textwidth}
      \begin{itemize}
        \item ``most complete AS-level topology''
        \item route servers, looking glasses, routing registries
        \item routing updates
      \end{itemize}
    \end{column}
    \begin{column}{0.08\textwidth}
      \begin{figure}
        \label{zhang_et_al}
        \includegraphics[width=1\textwidth]{images/zhang_b}\\
        \includegraphics[width=1\textwidth]{images/person}\\
        \includegraphics[width=1\textwidth]{images/massey}\\
        \includegraphics[width=1\textwidth]{images/zhang_l}
      \end{figure}
    \end{column}
    \begin{column}{0.3\textwidth}
      {\scriptsize Beichuan Zhang\\
      \vspace{0.1cm}
      UCLA\\
      \vspace{0.7cm}
      Raymond Liu\\
      \vspace{0.1cm}
      UCLA\\
      \vspace{0.3cm}
      Daniel Massey\\
      \vspace{0.1cm}
      Colorado State University\\
      \vspace{0.3cm}
      Lixia Zhang\\
      \vspace{0.1cm}
      UCLA\\ }
    \end{column}
  \end{columns}
\end{frame}


\begin{frame}{Modeling the Internet Routing Topology - In Less than 24h}{2009, \cite{Winter:2009:MIR:1577959.1577976}}
  \begin{columns}[c]
    \begin{column}{0.5\textwidth}
      \begin{itemize}
        \item gewichtete AS-Links
        \item AS-Graphen berechnen
        \item eine der Datenquellen für den Routingatlas
        \item Zugehöriges Projekt leider eingestellt
      \end{itemize}
    \end{column}
    \begin{column}{0.09\textwidth}
      \begin{figure}
        \label{winter}
        \includegraphics[width=1\textwidth]{images/winter_r}
      \end{figure}
    \end{column}
    \begin{column}{0.3\textwidth}
      {\scriptsize Rolf Winter\\
      \vspace{0.1cm}
      NEC Labs Europe\\ }
    \end{column}
  \end{columns}
\end{frame}

\begin{frame}{Nation-State Routing: Censorship, Wiretapping, and BGP}{2009, \cite{0903.3218v1}}
  \begin{columns}[c]
    \begin{column}{0.5\textwidth}
      \begin{itemize}
        \item Zuordnung IP-Präfix zu Land
        \item ...
      \end{itemize}
    \end{column}
    \begin{column}{0.09\textwidth}
      \begin{figure}
        \label{karlin}
        \includegraphics[width=1\textwidth]{images/karlin_j}\\
        \includegraphics[width=1\textwidth]{images/forrest_s}\\
        \includegraphics[width=1\textwidth]{images/rexford_j}
      \end{figure}
    \end{column}
    \begin{column}{0.3\textwidth}
      {\scriptsize Josh Karlin\\
      \vspace{0.1cm}
      University of New Mexico\\
      \vspace{0.7cm}
      Stephanie Forrest\\
      \vspace{0.1cm}
      University of New Mexico\\
      \vspace{0.5cm}
      Jennifer Rexford\\
      \vspace{0.1cm}
      Princeton University\\ }
    \end{column}
  \end{columns}
\end{frame}

\begin{frame}{IXPs: Mapped?}{2009, \cite{Augustin:2009:IM:1644893.1644934}}
  \begin{columns}[c]
    \begin{column}{0.5\textwidth}
      \begin{itemize}
        \item Präfixe und Mitglieder von IXPs finden
        \item traceroutes "durch" die IXPs
        \item (nicht propagierte) Peerings aufdecken
        \item AS-Graph vollständiger
      \end{itemize}
    \end{column}
    \begin{column}{0.1\textwidth}
      \begin{figure}
        \label{augustin}
        \includegraphics[width=1\textwidth]{images/augustin_b}\\
        \includegraphics[width=1\textwidth]{images/krishnamurthy_b}\\
        \includegraphics[width=1\textwidth]{images/willinger_w}
      \end{figure}
    \end{column}
    \begin{column}{0.3\textwidth}
      {\scriptsize Brice Augustin\\
      \vspace{0.1cm}
      Université Pierre et Marie Curie, Paris\\
      \vspace{0.5cm}
      Balachander Krishnamurthy\\
      \vspace{0.1cm}
      AT\&T Labs-Research, Floham Park\\
      \vspace{0.3cm}
      Walter Willinger\\
      \vspace{0.1cm}
      AT\&T Labs-Research, Floham Park\\ }
    \end{column}
  \end{columns}
\end{frame}

\section*{Agenda}
\begin{frame}{Agenda} \setcounter{tocdepth}{1} \tableofcontents[part=1] \setcounter{tocdepth}{3} \end{frame}

\section{Ausblick}
\begin{frame}{Ausblick}
\begin{itemize}
 \item Aktuelle shortest path matrix
 \begin{itemize}
  \item Topologieänderungen
 \end{itemize}\vspace{1cm}
 \item Andere Länder, Visualisierungen, Online-Tool, IPv6
 \begin{itemize}
  \item Breiteres Publikum
  \item Zukunftsfähigkeit
 \end{itemize}
\end{itemize}
\end{frame}

\part{Ende}
%\section{kthxbye}
\begin{frame}{Ende}
\begin{columns}[t]
\begin{column}{0.5\textwidth}
 \begin{center}
 \vspace{1cm}
 Vielen Dank für die Aufmerksamkeit\\
 \vspace{1.5cm}
 Fragen\ldots?
 \end{center}
\end{column}
\begin{column}{0.5\textwidth}
 \vspace{-1cm}
 \begin{figure}
  \label{asngraph_all}
%  \includegraphics[width=\textwidth]{asgraph_catall-pos-betweenness}
%  \caption{AS-Graph (gesamt DE)}
 \end{figure}
\end{column}
\end{columns}
\end{frame}

\section{Literatur}
\begin{frame}[plain, allowframebreaks]{Literatur}
\scriptsize
\bibliographystyle{alpha}
%\bibliographystyle{apalike}
\bibliography{folien}
\end{frame}

\section{Bildquellen}
\begin{frame}[plain]{Bilderquellen}
  %\scriptsize
  \tiny
  \begin{table}
    \begin{tabular}{ c p{0.8\textwidth} }
      Seite & Quelle \\ \hline
      & \multirow{3}{0.8\textwidth}{\url{http://www.cs.ucr.edu/~michalis/},\\ \url{http://www.cse.yorku.ca/cspeople/faculty/pfal/index.html},\\ \url{http://www.cs.cmu.edu/~christos/}} \\
      \pageref{faloutsos}& \\
      & \\ \hline
      \pageref{gao} & \url{http://www-unix.ecs.umass.edu/~lgao/} \\ \hline
      & \multirow{3}{0.8\textwidth}{\url{http://www.cs.arizona.edu/~bzhang/},\\ \url{http://www.cs.colostate.edu/~massey/},\\ \url{http://www.cs.ucla.edu/~lixia/}} \\
      \pageref{zhang_et_al} & \\
      & \\ \hline
      \pageref{winter} & \url{http://www.hs-augsburg.de/fakultaet/informatik/person/professor/winter_rolf/index.html} \\ \hline
      & \multirow{3}{0.8\textwidth}{\url{http://www.cs.unm.edu/~karlinjf/},\\ \url{http://www.cs.unm.edu/~forrest/},\\ \url{http://www.cs.princeton.edu/~jrex/}} \\
      \pageref{karlin} & \\
      & \\ \hline
      & \multirow{3}{0.8\textwidth}{\url{http://www-rp.lip6.fr/~augustin/},\\ \url{http://www.njit.edu/news/2011/2011-054.php},\\ \url{http://www.research.att.com/people/Willinger_Walter/index.html?fbid=Y-QjC_arIwn}} \\
      \pageref{augustin} & \\
      & \\ \hline
    \end{tabular}
  \end{table}
\end{frame}

\end{document}
